% -- Encoding UTF-8 without BOM
% -- XeLaTeX => PDF (BIBER)
\documentclass[]{cv-style}          % Add 'print' as an option into the square bracket to remove colours from this template for printing. 
                                    % Add 'espanol' as an option into the square bracket to change the date format of the Last Updated Text

\sethyphenation[variant=british]{english}{} % Add words between the {} to avoid them to be cut 
\usepackage{pifont}
\begin{document}

\header{Jacob}{Sandin}           % Your name
\lastupdated

%----------------------------------------------------------------------------------------
%	SIDEBAR SECTION  -- In the aside, each new line forces a line break
%----------------------------------------------------------------------------------------

\begin{aside}
%
\section{kontakt}
Hemmeslövsvägen 40
269 96 Båstad
Sverige
~
+46 (0)76 208 33 94
~
jacob@js.se
%
\section{språk}
Svenska modersmål
Engelska flytande
Latin skoj men glömt
%
\section{programming}
{\color{red} $\varheartsuit$} Pascal, {\color{red} $\heartsuit$} Rust, Java, C\#, SQL, Bash, {\color{orange} \ding{72}}Ansible
\LaTeX{}, mm.
%
\end{aside}

%----------------------------------------------------------------------------------------
%	SKILLS SECTION
%----------------------------------------------------------------------------------------

\section{färdighet}
  \vspace{-0.2cm}

  Linux, Windows, Programmering, System management, Networking, TCP, Shell, Vim, Support, Management, System planering, KOHA, VuFind, SOLR
%----------------------------------------------------------------------------------------
%	WORK EXPERIENCE SECTION
%----------------------------------------------------------------------------------------

\section{erfarenhet}

\begin{entrylist}
%------------------------------------------------
\entry
  {2020--Now}
  {imCode Partner AB}
  {Visby, Sverige}
  {\jobtitle{System utvecklare}\\
      ImCode jobbar i första hand med olika ASP lösningar som är fokuserade
      mot större organisationer, kommuner och landsting. Speciellt då avancerade
      CMS system och biblioteks system. Som systemutvecklare jobbar jag med
      båda dessa inriktningar, mestadels bakom kulisserna. Mit arbete består
      här med att se till att systemen bakom fungerar; Linux-, SQL-, 
      Indexerings-servrar och liknande. Men även utveckling av mjukvara för
      denna drift och nya funktioner för kunderna.

%    Arbetsuppgifter:
%  \begin{itemize}
%    \item Systemutveckling
%    \item Virtualisering av våra egna och kundmiljöer.
%    \item Automatisering av system, mestadels med Ansible,
%    \item Installationer av ny och känd mjukvara.
%    \item Guidning och testning åt andra utvecklare.
%    \item Driftövervakning av egna och kundsystem.
%    \item Kundsupport gällande mer tekniska frågor.
%    \item Internsupport gällande tekniska planeringar och frågor.
%    \item Utveckling, installation och underhåll för KOHA,Lots och VuFind.
%    \item Utveckling, installation och underhåll Lots VuFind
 %   \item ...
 %   \item (detta var några av mina uppgifter.)

  %\end{itemize}
  }
%------------------------------------------------
\entry
  {2009-2020}
  {Cefit AB}
  {Båstad, Sverige}
  {\jobtitle{Ägare}\\
      I Cefit AB (Centrum för effektiv IT) så jobbade jag mestadels som 
      ISP (Internettjänster) och ASP (Applikations tjänster). Exempelvis
      brandväggs tjänster, serverhotell och tekniskt stöd.

      Större delen av mina kunder var erfarna förmedlare av ovan tjänster
      själva, inom kommun, landsting och företag. Ofta datoravdelningar 
      här som behövde hjälp med de mer tekniska lösningar och funktioner.
      
%    Arbetsuppgifter:
%  \begin{itemize}
%    \item Achievement 1. Achievement 1. Achievement 1. 
%    \item Achievement 2. Achievement 2. Achievement 2. Achievement 2. Achievement 2. Achievement 2.
%    \item Achievement 3. Achievement 3. Achievement 3. Achievement 3.  
%  \end{itemize}}\\
}   
%------------------------------------------------
\entry
  {1999--2011}
  {Sverige.net / It-Halland (Media network AB)}
  {Halmstad, Sverige}
  {\jobtitle{IT Chef}\\
    I mitt arbete här började jag som systemutvecklare och i supporten, och även delvis
    som programmerare då man körde ett order/kund/fakturerings system som jag byggt 
    åt Buller Data tidigare.

    Som it-chef och verksamhets chef för internet leverantörsbiten så utvecklade
    jag denna verksamhet från en 100tkr verksamhet till en mångmiljonverksamhet.
    Givetvis inte själv utan jag hade duktiga medarbetare och det låg mycket i
    tiden så at säga. Samtidigt var det få som hade den kompetens jag hade.
    Faktum var att de inte hittade någon annan när jag anställdes utan köpte
    loss mig från en annan sysselsättning.

    När jag slutade här var vi en av de största aktörerna i området. Samtidigt
    som nybliven pappa upplevde jag att jag aldrig fick se mina barn.

%    Arbetsuppgifter:
%  \begin{itemize}
%    \item Achievement 1. Achievement 1. Achievement 1. Achievement 1. Achievement 1. Achievement 1. Achievement 1. Achievement 1. 
%  \end{itemize}
} 
%------------------------------------------------
\entry
  {1993--1999}
  {Buller Data AB}
  {Båstad, Sverige}
  {\jobtitle{Job Title}\\
    Buller data var en pionjär inom dator och IT i Sverige och det var en
    oerhörd fördel för mig att få börja min IT karriär här och i denna 
    miljö. En miljö av sann skaparglädje, man ville verkligen skapa något
    nytt med den nya tekniken och de möjligheter denna gav. Jag fick här
    tillgång till mentorer som var några av de skarpaste inom svensk utveckling
    då. 

    När jag började på Buller Data så började jag som praktikant på lagret (via 
    arbetsförmedlingen). Jag blev samma år ansvarig för lagret, Buller Data var
    då en av Sveriges största postorder företag för datordelar, CD och CD-rom.
    Vi hade också en av världens största BBS:er med användare från hela världen,
    där jag var administratör.

    1993 startade vi med Internettjänster och var därmed en av de absolut första 
    internetleverantörerna för internet via modem till privatpersoner i Sverige.
    Jag var en av nyckelpersonerna i att utveckla denna verksamhet och det gav
    mig också möjligheten att ta upp programmering igen.

%    Arbetsuppgifter:
%  \begin{itemize}
%    \item Achievement 1. Achievement 1. Achievement 1. Achievement 1. Achievement 1. Achievement 1. Achievement 1. Achievement 1. 
%  \end{itemize}
}
\pagebreak
%------------------------------------------------

\end{entrylist}

%----------------------------------------------------------------------------------------
%	EDUCATION SECTION
%----------------------------------------------------------------------------------------

\section{bakgrund}
Jag är född i Malmö 1970 och har levt större delen av mitt liv i södra Sverige.
Med undantag av några år i Uppsala, Halmstad och UAE (Förenade Arabemiraten).

Vid sex års ålder fick jag cancer i mitt vänstra ben och vart amputerad. Detta
innebär inga större problem för mig utan har nog bidragit till en målmedvetenhet
och kanske envishet. Samtidigt undviker jag att stå långa tider eller att bära
tunga saker längre sträckor.

I 12-13 års åldern bodde jag i UAE, både ABU-Dhabi och Dubai, vilket har
bidragit till en god förståelse för olika kulturer och öppenhet inför dessa.
Samtidigt som min skolgång i Engelsktalande skolor bidrog till en väl utvecklad
Engelska, även med tänkande i Engelska och flytande tal. Det senare kanske
kräver lite exponering numer innan det kickar igång.

Som barn var jag väldigt praktisk och jobbade extra på bil-verkstäder, skroten
och billackeraren i hembyn. Intresset var nog mer hur saker funkar än just
bilar. Samtidigt var det det som fans att tillgå.

Önskan att förstå hur saker fungerade var också vad som fick mig att söka
mig till lantbruk. Det var fritt, frisk luft och mycket "hur funkar det".
Därför är detta den utbildning jag har i grunden och varför jag inte 
dokumenterat utbildning.

När det gäller datorer, är jag i princip helt självlärd även om jag haft
tur med praktik och att få kontakt med duktiga människor och förebilder
inom just datorer/IT. Min första dator var mina föräldrars dator de köpte
för sin bokföring, en IBM PC 1982 ett år efter lanseringen i New York 1981.
I strax efter vilket jag tog mina första stapplande steg med programmering,
men blev allvar först som anställd på Buller Data långt senare.

Jag har även kortare utbildningar och erfarenhet i Media samt verbal 
kommunikation Klassiskt klarspråk (Noneviolent communication) även kallat
Giraff-språket inom militären.

Givetvis har jag många historier och detaljer, från ett långt och 
händelserikt liv, men inte just här.
\begin{entrylist}
%------------------------------------------------
\entry
{1986--1988}
{Lantbruks Skolan {\normalfont Husdjur, Bygg, Företagsekonomi, Maskin, Växtlära }}
{Gymnasie}
{\vspace{-0.80cm}}
%------------------------------------------------
\entry
{1990}
{Media och Radio {\normalfont Journalistik, Radiopratare, Producering, 
Ljudhantering}}
{Gymnasie}
{\vspace{-0.8cm}}
%------------------------------------------------
\entry
{1990 -- }
{Klassiskt klarspråk {\normalfont Detta är en ständigt pågående utbildning}}
{Adito bildning AB}
{Träningar någon gång per år.}
{\vspace{-0.5cm}}
%------------------------------------------------
\end{entrylist}

%----------------------------------------------------------------------------------------
%	AWARDS SECTION
%----------------------------------------------------------------------------------------

\section{egenheter}

\begin{entrylist}
%------------------------------------------------
\entry
{\color{green} \ding{52}}
{Fokuserad}
{Envis}
{Att vara fokuserad (eller envis) är något som är både bra och dåligt. En bra 
sak är att man kan lösa väldigt komplexa problem om man bara har 
möjligheten att fokusera. En mindre bra är att jag kan få svårt om jag har
många störande moment.}
{\vspace{-0.8cm}}
%------------------------------------------------
\entry
{\color{green} \ding{52}}
{Lösningsorienterad}
{}
{Jag funderade på att skriva kundorienterad, samtidigt så är det två sidor
av samma mynt. Det är ändå för att kunderna skall bli de bästa de kan i sitt
arbete som vi gör det mesta vi gör.}
{\vspace{-0.4cm}}
%------------------------------------------------
\entry
{\color{green} \ding{52}}
{Ansvarsfull}
{Ibland till överdrift}
{Jag tycker verkligen inte om att lämna saker ogjort, eller att känna att någon
väntar på mig att göra klart. Jag kan jobba dygnet runt för att inte hamna i
den situationen, något som är problematiskt efter som man ofta får mer jobb då.
Tills det blir för mycket, och man inte hinner med.}
{\vspace{-0.4cm}}
%------------------------------------------------
\entry
{\color{green} \ding{52}}
{Självgående}
{}
{Jag har oftast lätt att se sådant som behövs ta hand om, behov i den
egna organisationen såväl som hos kunder jag har kontakt med. Samtidigt
får jag ofta förtroendet att ta hand om just sådant. Investeringar,
dokumentation, planering.}
{\vspace{-0.5cm}}
%------------------------------------------------
\end{entrylist}

%----------------------------------------------------------------------------------------
%	OTHER QUALIFICATIONS SECTION
%----------------------------------------------------------------------------------------
%
%\section{education}
%
%\begin{entrylist}
%%------------------------------------------------
%\entry
%{2013}
%{Qualification}
%{Institution}
%{\vspace{-0.3cm}}
%%------------------------------------------------
%\entry
%{2011}
%{Qualification}
%{Institution}
%{\vspace{-0.3cm}}
%%------------------------------------------------
%\end{entrylist}

%----------------------------------------------------------------------------------------
%	INTERESTS SECTION
%----------------------------------------------------------------------------------------
\section{interests}
  \vspace{-0.2cm}

  \textbf{professional:} Programmering, Linux, Mjukvara, Nya tekniker, 
  Paradigmer (programmering).
  Över huvud taget att lära mig nya saker och utvecklas.\\
\textbf{personal:} Kajak, Friluftliv, Fiske, Natur, Rustikt leverne(bushcraft), 
Växter, Djurliv. Jag läser mycket(ljudbok) oftast fakta normalt 50+ böcker/år

%----------------------------------------------------------------------------------------

\end{document}
